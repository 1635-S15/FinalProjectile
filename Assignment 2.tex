\newcommand{\sn}{Syler Wagner}					% name		right header
\newcommand{\assnum}{2}							% assignment number				
\newcommand{\sa}{16.35 Assignment \#\assnum } 	% title 	left header
%\newcommand{\sa}{24.903 Problem Set \assnum} 	% title 	left header
\newcommand{\sd}{Spring 2015}					% bottom left footer

\documentclass[12pt]{article}
% \setlength{\parindent}{0pt} 		%% NO INDENT
% \setlength{\parskip}{1em} 		%% PARAGRAPH SKIP
%% TOGGLE TITLE PAGE__________________________________________________________________________
	\newcommand{\titlepagee}{\title{\sa}	%% TOGGLE TITLE PAGE % TITLE
						 \author{\it }%Requirements Workshop }				% NAME 
						 \date{ }				% DATE
						 \maketitle} 
%% TOGGLE PROBLEM STATEMENT NAME______________________________________________________________
	\newcommand{\psetproblem}{Problem }
	%\newcommand{\psetproblem}{Part }
	%\newcommand{\psetproblem}{Question }
%% TOGGLE CHEATSHEET__________________________________________________________________________
	%\documentclass[6pt]{article}
	%\usepackage[includeheadfoot,margin=0.5in,foot=75pt]{geometry} 
	%\newcommand{\docstart}{\begin{multicols*}{2} \setlength{\parindent}{0pt}}
	%\newcommand{\docend}{\end{multicols*}}

% \setlength{\parindent}{0pt} 		%% NO INDENT
% \setlength{\parskip}{1em} 		%% PARAGRAPH SKIP

%%%%%%%%%%%%%%%%%%%%%%%%%%%%%%%%%%%%%%%%%%%%%%%%%%%%%%%%%%%%%%%%%%%%%%%%%%%%%%%%%%%%%%%%%%%%%%
	\usepackage{sytex} 
	\usepackage{enumitem}

\newlist{legal}{enumerate}{10}
\setlist[legal]{label*=\arabic*.}
	%\usepackage{linguex}
	\usepackage{listings}
\usepackage{color}

\definecolor{dkgreen}{rgb}{0,0.6,0}
\definecolor{gray}{rgb}{0.5,0.5,0.5}
\definecolor{mauve}{rgb}{0.58,0,0.82}

\lstset{frame=tb,
  language=Java,
  aboveskip=3mm,
  belowskip=3mm,
  showstringspaces=false,
  columns=flexible,
  basicstyle={\footnotesize\ttfamily},
  numbers=none,
  numberstyle=\tiny\color{gray},
  keywordstyle=\color{blue},
  commentstyle=\color{dkgreen},
  stringstyle=\color{mauve},
  breaklines=true,
  breakatwhitespace=true,
  tabsize=3
}
%%%%%%%%%%%%%%%%%%%%%%%%%%%%%%%%%%%%%%%%%%%%%%%%%%%%%%%%%%%%%%%%%%%%%%%%%%%%%%%%%%%%%%%%%%%%%%
%% Redefining sections as problems
	\makeatletter
	\newenvironment{problem}{\@startsection
	       {section}
	       {1}
	       {-.2em}
	       {-3.5ex plus -1ex minus -.2ex}
	       {2.3ex plus .2ex}
	       {\pagebreak[3]%forces pagebreak when space is small; use \eject for better results
	       \large\bf\noindent{\psetproblem }
	       }
	       }
	       {%\vspace{1ex}\begin{center} \rule{0.3\linewidth}{.3pt}\end{center}}
	       \begin{center}\large\bf \ldots\ldots\ldots\end{center}}
	\makeatother
	\makeatletter
	\newcounter{subsubparagraph}[subparagraph]
	\renewcommand\thesubsubparagraph{%
	  \thesubparagraph.\@arabic\c@subsubparagraph}
	\newcommand\subsubparagraph{%
	  \@startsection{subsubparagraph}    % counter
	    {6}                              % level
	    {7.5ex}                     % indent
	    {} % beforeskip
	    {}                           % afterskip
	    {\normalfont\normalsize\bfseries}}
	\newcommand\l@subsubparagraph{\@dottedtocline{6}{10em}{5em}}
	\newcommand{\subsubparagraphmark}[1]{}
	\makeatother
	%Fancy-header package to modify header/page numbering 
	\usepackage{fancyhdr}
	\pagestyle{fancy}
	\lfoot{\small\scshape \sd}
	\chead{} 
	\rfoot{\thepage} 
	\lhead{\small\scshape \sa} 
	\cfoot{} 
	\rhead{\small\scshape \sn} 
	\renewcommand{\headrulewidth}{.3pt} 
	\renewcommand{\footrulewidth}{.3pt}
	\setlength\voffset{-0.25in}
	\setlength\hoffset{-0.18in}
	\setlength\textheight{648pt}
%%%%%%%%%%%%%%%%%%%%%%%%%%%%%%%%%%%%%%%%%%%%%%%%%%%%%%%%%%%%%%%%%%%%%%%%%%%%%%%%%%%%%%%%%%%%%%
%%%%%%%%%%%%%%%%%%%%%%%%%%%%%%%%%%%%%%%%%%%%%%%%%%%%%%%%%%%%%%%%%%%%%%%%%%%%%%%%%%%%%%%%%%%%%%
\begin{document}
\titlepagee
\thispagestyle{fancy} 	% TOGGLE HEADER FOOTER ON FIRST
\docstart 				% this is broken unless it's a cheatsheet
%  _ 
% / |
% | |
% |_| 
%% PROBLEM 1 

%\ble {\tt GroundVehicle}
%\ble Variables
%	\ble Pose {\tt pose[3]=(}$x,y,\h${\tt )}
%		\ble $x$ shall be in the interval $[0,100]$
%		\dl $y$ shall be in the interval $[0,100]$
%		\dl $\h$ shall be in the interval $[-\pi,\pi)$
%	\ele 
%	\dl \rrr{Velocity {\tt vel[3]=(}$\px,\p{y},\p{\h}${\tt )}
%		\ble $\rr{\px + \p{y}}$ shall be in the interval $[5,10]$
%		\dl $\p{\h}$ shall be in the interval $[-\pi/4,\pi/4]$}
%	\ele
%\ele
%\dl Constructor
%\ble {\tt GroundVehicle(double pose[3], double s, double omega)}
%	\ble $s$ shall be in the interval $[5,10]$ m/sec
%	\dl $\w$ shall be in the interval $[-\pi,\pi]$ rad/s
%
%\ele\ele
%\dl Methods
%\ble
%	{\tt double [] getPosition()}  shall return an array of 3 doubles, corresponding to the $x, y, \h$ position and orientation of the vehicle.
%	\dl {\tt double [] getVelocity()} shall return an array of 3 doubles, corresponding to the $\px, \p{y}, \p{\h}$ linear and angular velocities of the vehicle.
%	\dl {\tt setPosition(double [3])} 
%	\ble This method shall take an array of 3 doubles, corresponding to the $x, y, \h$ position of the vehicle, and sets the corresponding internal representation of the vehicle position.
%	\dl If the {\tt setPosition} method attempts to exceed the position constraints, the position shall be clamped to the position within the limits that is closest to the desired position.\ele
%	\dl {\tt setVelocity(double [3])}
%	\ble This method takes an array of 3 doubles, corresponding to the $\px, \p{y}, \p{\h}$ linear and angular velocities of the vehicle, and sets the corresponding internal representation of the vehicle velocities.
%\dl If the {\tt setVelocity} method attempts to exceed the velocity constraints, the position shall be clamped at the nearest limit, \rrr{where in the linear velocities, the nearest velocity is the closest velocity in Euclidean distance in velocity space.}\ele
%	\dl {\tt controlVehicle(Control c)}
%	\ble This method modifies the internal velocities according to the specified forward speed and rotational velocity.
%	\dl \rrr{ The velocities that result from applying the control must obey the same limits as setVelocity.}\ele
%	\dl {\tt updateState(int sec, int usec)} \rrr{changes the vehicle internal state by computing the appropriate kinematic and dynamic change that would occur after time $t$, where $t$ is given in two arguments: seconds and milliseconds.}
%	\dl {\tt }
%\ele
%\ele


%\ble 
\Big{\bb{2. {\tt  GroundVehicle}  Requirements}}


\ble  \label{itm:first} Variables
		\ble \label{itm:p1} The {\tt GroundVehicle} shall contain the internal representations of $x$ and $y$ position in two-dimensional space
		\ble \label{xlim} $x$ shall be in the interval $[0,100]$
		\dl \label{ylim} $y$ shall be in the interval $[0,100]$\ele
	\dl The {\tt GroundVehicle} shall contain the internal representation of a heading angle $\h$
		\ble $\h$ will describe the the {\tt GroundVehicle}'s orientation in two-dimensional space
		\dl \label{itm:p2} $\h$ shall be in the interval $[-\pi,\pi)$
		\ele
\dl The {\tt GroundVehicle} shall contain  internal representations of the linear velocities $\px$ and $\p{y}$		
		\ble \label{itm:v1}  $\rr{\px + \p{y}}$ shall be in the interval $[5,10]$\ele
	\dl \label{hdot} The {\tt GroundVehicle} shall contain the internal representation of an angular velocity $\p{\h}$ 
		\ble \label{itm:v2} $\p{\h}$ shall be in the interval $[-\pi/4,\pi/4]$}
	\ele
\ele
\dl Constructor
\ble  \label{itm:2} {\tt GroundVehicle(double pose[3], double dx, double dy, double dtheta)}
	\ble The constructor shall take four arguments
		\ble An {\tt IllegalArgumentException} shall be thrown if the {\tt double pose[3]} argument is not of length 3\ele
		\ble An {\tt IllegalArgumentException} shall be thrown if any of the arguments' types differ from those specified in \ref{itm:2}\ele
	\dl The internal representation of the $x, y, \h$ pose, and $\px, \p{y}, \p{\h}$ velocities shall be initialized according to the constructor arguments
	\dl The constructor shall initialize the variables to values within the intervals specified in \ref{itm:p1}-\ref{hdot} 
		\ble \label{itm:clampp} 
		If an element in the array {\tt pose[3]} $= [x,y,\h]$ falls outside the allowable interval specified in \ref{xlim}, \ref{ylim} and \ref{itm:p2}, the internal representation of that position variable shall be clamped at the nearest limit
		\ble If the value falls below the specified range, the internal representation of that position variable shall be initialized to the lower limit of the allowable interval.
		\dl If the value falls above the specified range, the internal representation of that position variable shall be initialized to the upper limit of the allowable interval \ele

				\dl \label{itm:clamp} If a velocity value falls outside the allowable intervals specified in \ref{itm:v1} and \ref{itm:v2}, the internal representation of that velocity shall be clamped at the nearest limit
					\ble If the value falls below the specified range, the internal representation of that velocity shall be initialized to the lower limit of the allowable interval
		\dl If the value falls above the specified range, the internal representation of that velocity shall be initialized to the upper limit of the allowable interval \ele

\dle 

		\ele
	\ele
\dl \label{itm:sw} {\tt GroundVehicle(double pose[3], double s, double omega)}
	\ble The constructor shall take three arguments
		\ble An {\tt IllegalArgumentException} shall be thrown if the first argument is not an array of length 3\ele
		\ble An {\tt IllegalArgumentException} shall be thrown if any of the arguments' types differ from those specified in \ref{itm:sw}\ele
	\dl The internal representation of $x, y, \h, \px, \p{y}, \p{\h}$ shall be initialized according to the constructor arguments
		\ble The position and heading shall be initialized to the values of the array {\tt pose[3]} $= [x,y,\h]$
		\dl \label{som} The linear velocities shall be calculated based on forward speed $s$ and heading angle $\h$ according to equations \ref{cos} and \ref{sin}
		\ble \label{cos} $\px = s\cd\cos(\h)$ 
		\dl \label{sin} $\p{y} = s\cd\sin(\h)$
		\ele
		\dl \label{sow} The internal representation of angular velocity shall be initialized to the value $\p{\h} =$ {\tt omega}
		\ele
	\dl The constructor shall initialize the variables to values within the intervals specified in \ref{itm:p1}-\ref{hdot} 
		\ble  If an element in the array {\tt pose[3]} $= [x,y,\h]$ falls outside the allowable interval specified in \ref{xlim}, \ref{ylim} and \ref{itm:p2}, the internal representation of that position variable shall be clamped at the nearest limit as described in \ref{itm:clampp}
				\dl  If a velocity value falls outside the allowable intervals specified in \ref{itm:v1} and \ref{itm:v2}, the internal representation of that velocity shall be clamped at the nearest limit as described in \ref{itm:clamp}

\ele\ele
\dl Methods
\ble
	{\tt double [] getPosition()} 
	\ble This method shall return an array of 3 doubles, corresponding to the $x, y, \h$ position and orientation of the vehicle\ele
	\dl {\tt double [] getVelocity()} 
	\ble This method shall return an array of 3 doubles, corresponding to the $\px, \p{y}, \p{\h}$ linear and angular velocities of the vehicle\ele
	\dl {\tt setPosition(double [3])} 
	\ble This method shall take one argument, an array of 3 doubles, corresponding to the $x, y, \h$ position of the vehicle.
	\ble  An {\tt IllegalArgumentException} shall be thrown if the argument is not an array of length 3\ele
	\dl The {\tt setPosition} method shall set the internal representation of the vehicle position according to the values contained within the argument array
	\dl If the {\tt setPosition} method attempts to exceed the position constraints in  \ref{itm:p1} and  \ref{itm:p2}, the position shall be clamped as described in \ref{itm:clampp}	\ele
	
	\dl {\tt setVelocity(double [3])}
	\ble This method shall take one argument, an array of 3 doubles corresponding to the $\px, \p{y}, \p{\h}$ linear and angular velocities of the vehicle
		\ble  An {\tt IllegalArgumentException} shall be thrown if the argument is not an array of length 3\ele
	\dl The {\tt setVelocity} method shall set the internal representation of the vehicle velocities according to the values contained within the argument array
\dl \label{svel} If the {\tt setVelocity} method attempts to exceed the velocity constraints in  \ref{itm:v1} and  \ref{itm:v2}, the velocity shall be clamped as described in \ref{itm:clamp}	\ele
	\dl {\tt controlVehicle(Control c)}
	\ble This method shall modify the internal representations of the $\px, \p{y}, \p{\h}$ velocities according to the specified forward speed and rotational velocity in the {\tt Control c} argument.
			\ble $\px$ and $\p{y}$ shall be calculated based on forward speed $s$ and heading angle $\h$ as described in \ref{som}		
		\dl $\p{\h}$ shall be initialized to the value of rotational velocity $\w$
		\ele
	\dl  The internal representations of the velocities that result from applying the control must obey the same limits as {\tt setVelocity} in \ref{svel}\ble If the {\tt controlVehicle} method attempts to exceed the velocity constraints in  \ref{itm:v1} and  \ref{itm:v2}, the velocity shall be clamped as described in \ref{itm:clamp}	\ele\ele
	\dl {\tt updateState(int sec, int usec)}
		\ble  This method shall take two arguments: the seconds {\tt (sec) }and milliseconds {\tt (usec)} comprising $t$}
	\ble The arguments shall be combined according to the equation \ref{timeq}, where $t[s] =$ {\tt sec} and $t[\mu s] =$ {\tt usec} 
	\ble \label{timeq} $t=t[s]+t[\mu s]\cd 10^{-3}$  
	\dl The resulting $t$ value will represent the time in seconds\ele\ele
	\dl The {\tt updateState} method will change the vehicle internal state by computing the appropriate kinematic and dynamic change that would occur after time $t$
	\ble The {\tt updateState} method shall change the internal representation of the $x, y, \h$ pose, and $\px, \p{y}, \p{\h}$ velocities according to the dynamics calculated.\ele\ele\ele\ele\ele\ele
	
	
	
	\xxx
	
	\Big{\bb{3.} {$ $ }{\tt  Simulator}{ $ $} {\it }}
	

\begin{figure}[H]
  \centering
\icc{scale=0.45}{s1}
  \caption{{\tt DisplayServer} window showing the star-shaped path produced by a single vehicle}
\end{figure}

	\Big{\bb{4.  Mutexes and Synchronization}}
	
In terms of mutual exclusion and critical regions, there are a number of shared resources, and a number of conditions on entry into critical regions. As a pre-deliverable, please identify the shared resources in every class. Please also identify the lock either by variable name or class name that is used to protect each shared resource.
For any conditional critical regions, please identify the conditional critical regions (you can do this colloquially, i.e., changing variable X) and identify what conditions must be satisfied for a thread to enter the conditional region.
You must include in your submission:
\qq{1.} $[$PDF$]$ In your pre-deliverable, please list the shared resources and the names of the variables or classes used to protect these resources.\qe

\begin{lstlisting}
public class Simulator {
    /* Shared resource */
    protected int _vehicleControlQueue = 0;     // number of GroundVehicles waiting for controls
    private int _currentSec = 0;    /* shared resource */
    private int _currentMSec = 0;   /* shared resource */
\end{lstlisting}

\begin{lstlisting}
public class GroundVehicle {
	private double _x, _y, _theta;    /* Shared resources */
	private double _dx, _dy, _dtheta; /* Shared resources */
\end{lstlisting}

\qq{2.} $[$PDF$]$ In your pre-deliverable, please identify the conditional critical regions and what the conditions are for thread entry.\qe
\begin{lstlisting}
public class VehicleController extends Thread {
	public void run() {
        int currentSec = 0;
        int currentMSec = 0;
        while (currentSec < 100) { // while simulator time is less than 100 seconds
            synchronized (_s) { /* Conditional critical region */
                // get the current simulator time
                currentSec = _s.getCurrentSec();
                currentMSec = _s.getCurrentMSec();
                // if the simulator time hasn't changed since it was last checked
                while (_lastCheckedSec == currentSec && _lastCheckedMSec == currentMSec) {
                    try {
                        // simulator waits for controllers to update
                        _s.wait();
                        currentSec = _s.getCurrentSec();
                        currentMSec = _s.getCurrentMSec();
                    } catch (InterruptedException e) {
                        e.printStackTrace();
                    }
                }
            }
            // if the simulator time has changed since it was last checked
            // update last checked time
            _lastCheckedSec = currentSec;
            _lastCheckedMSec = currentMSec;
            // get ground vehicle state
            _v.getPosition();
            _v.getVelocity();
            // get next control and apply control to GroundVehicle
            Control c = getControl(currentSec, currentMSec);
            _v.controlVehicle(c);
            synchronized (_s) { /* Conditional critical region */
                // if there were vehicles queued for controls
                if (_s._vehicleControlQueue > 0) {
                    // decrement number of vehicles waiting for controls
                    int prevQueueValue = _s._vehicleControlQueue;
                    _s._vehicleControlQueue--;
                    // if there are no vehicles queued for controls
                    if (_s._vehicleControlQueue == 0) {
                        // notify all waiting threads
                        _s.notifyAll();
                    }
                } else if (_s._vehicleControlQueue < 0) {
                    throw new IllegalStateException("Number of vehicles waiting for controls is negative.");
                }
            }
        } // end-while
    }
\end{lstlisting}



	\Big{\bb{5. More interesting {\tt   VehicleController}s}}

\begin{figure}[H]
  \centering
\icc{scale=0.45}{s2}
  \caption{{\tt DisplayServer} showing the trajectory of the {\tt RandomController} running by itself}
\end{figure}
\begin{figure}[H]
  \centering
\icc{scale=0.45}{s3}
  \caption{{\tt DisplayServer} running one vehicle with the {\tt RandomController} and two vehicles running the {\tt FollowingController}}
\end{figure}

	\Big{\bb{6. Time Spent on Deliverables}}\\
	
	\medskip
		
\noindent		
	   {\tt GroundVehicle}: 2.2 hours
		
\noindent		
	   {\tt     Simulator}: 13.9 hours
	
\noindent		
		   {\tt     VehicleController}: 3.8  hours
	
\noindent		
	   {\tt     RandomController}: 1.4 hours
	
\noindent		
	   {\tt     FollowingController}: 9.0 hours
	
\noindent		
	   {\tt     TestGroundVehicle}: 2.7 hours
	
\noindent		
	   {\tt     TestSimulator}: 2.3 hours
	
\noindent		
	   {\tt     TestVehicleController}: 3.0 hours\\
	
		   {\tt     GroundVehicle} Requirements: 6.1 hours
	
	     PDF Writeup: 2.6 hours
	
	     Miscellaneous: 4.4 hours\\


\noindent		
	      Total: 51.3 hours
	
	
%\xxx
%\ele
%\ele
%\dl The {\tt VehicleController} - where do I day it inherits from Thread?
%\ble
%	\dl Variables
%	\ble
%	\dl
%\ele
%	\dl Constructor
%	\ble
%		\dl {\tt public VehicleController(Simulator s, GroundVehicle v)}
%	\ele
%	\dl Methods
%	\ble
%		\dl {\tt \rrr{old stuff}}
%		\dl {\tt public void run()}
%		\dl {\tt setNumSides(int n)}
%		\dl {\tt addGroundVehicle(GroundVehicle gv)}
%		\dl {\tt }
%		\dl {\tt }
%		\dl {\tt }
%		\dl {\tt }
%	\ele
%\ele
%\dl {\tt Simulator}
%\ble
%	\dl Variables
%	\ble
%	\dl Instead of user-entry IP, get it automatically - look into DisplayClient class
%\ele
%	\dl Constructor
%	\ble
%		\dl {\tt }
%	\ele
%	\dl Methods
%	\ble
%		\dl {\tt \rrr{old stuff}}
%		\dl {\tt run()}
%		\dl {\tt main}
%		\dl {\tt addGroundVehicle(GroundVehicle gv)}
%		\dl {\tt }
%		\dl {\tt }
%		\dl {\tt }
%		\dl {\tt }
%	\ele
%\ele
%\ele
%-------------------------------------------------------------------------------%
%
%
%\dl {\tt GroundVehicle}
%\ble
%	\dl Variables
%	\ble
%	\dl
%\ele
%	\dl Constructor
%	\ble
%		\dl {\tt GroundVehicle(double pose[3], double dx, double dy, double dtheta)}
%	\ele
%	\dl Methods
%	\ble
%		\dl {\tt double [] getPosition()}
%		\dl {\tt double [] getVelocity()}
%		\dl {\tt setPosition(double [3])}
%		\dl {\tt setVelocity(double [3])}
%		\dl {\tt controlVehicle(Control c)}
%		\dl {\tt updateState(int sec, int usec)}
%		\dl {\tt }
%	\ele
%\ele
%\dl The {\tt VehicleController} - where do I day it inherits from Thread?
%\ble
%	\dl Variables
%	\ble
%	\dl
%\ele
%	\dl Constructor
%	\ble
%		\dl {\tt public VehicleController(Simulator s, GroundVehicle v)}
%	\ele
%	\dl Methods
%	\ble
%		\dl {\tt \rrr{old stuff}}
%		\dl {\tt public void run()}
%		\dl {\tt setNumSides(int n)}
%		\dl {\tt addGroundVehicle(GroundVehicle gv)}
%		\dl {\tt }
%		\dl {\tt }
%		\dl {\tt }
%		\dl {\tt }
%	\ele
%\ele
%\dl {\tt Simulator}
%\ble
%	\dl Variables
%	\ble
%	\dl Instead of user-entry IP, get it automatically - look into DisplayClient class
%\ele
%	\dl Constructor
%	\ble
%		\dl {\tt }
%	\ele
%	\dl Methods
%	\ble
%		\dl {\tt \rrr{old stuff}}
%		\dl {\tt run()}
%		\dl {\tt main}
%		\dl {\tt addGroundVehicle(GroundVehicle gv)}
%		\dl {\tt }
%		\dl {\tt }
%		\dl {\tt }
%		\dl {\tt }
%	\ele
%\ele
%\ele
%
%\begin{problem}{\it {\tt GroundVehicle} Requirements (5 pts)}
%
%
%\ble
%\dl {\tt GroundVehicle}
%\ble
%	\dl Variables
%	\ble
%	\dl
%\ele
%	\dl Constructor
%	\ble
%		\dl {\tt GroundVehicle(double pose[3], double dx, double dy, double dtheta)}
%	\ele
%	\dl Methods
%	\ble
%		\dl {\tt double [] getPosition()}
%		\dl {\tt double [] getVelocity()}
%		\dl {\tt setPosition(double [3])}
%		\dl {\tt setVelocity(double [3])}
%		\dl {\tt controlVehicle(Control c)}
%		\dl {\tt updateState(int sec, int usec)}
%		\dl {\tt }
%	\ele
%\ele
%\dl The {\tt VehicleController} - where do I day it inherits from Thread?
%\ble
%	\dl Variables
%	\ble
%	\dl
%\ele
%	\dl Constructor
%	\ble
%		\dl {\tt public VehicleController(Simulator s, GroundVehicle v)}
%	\ele
%	\dl Methods
%	\ble
%		\dl {\tt \rrr{old stuff}}
%		\dl {\tt public void run()}
%		\dl {\tt setNumSides(int n)}
%		\dl {\tt addGroundVehicle(GroundVehicle gv)}
%		\dl {\tt }
%		\dl {\tt }
%		\dl {\tt }
%		\dl {\tt }
%	\ele
%\ele
%\dl {\tt Simulator}
%\ble
%	\dl Variables
%	\ble
%	\dl Instead of user-entry IP, get it automatically - look into DisplayClient class
%\ele
%	\dl Constructor
%	\ble
%		\dl {\tt }
%	\ele
%	\dl Methods
%	\ble
%		\dl {\tt \rrr{old stuff}}
%		\dl {\tt run()}
%		\dl {\tt main}
%		\dl {\tt addGroundVehicle(GroundVehicle gv)}
%		\dl {\tt }
%		\dl {\tt }
%		\dl {\tt }
%		\dl {\tt }
%	\ele
%\ele
%\ele
%  ___ 
% |_  )
%  / / 
% /___|
%% PROBLEM 2 %-------------------------------------------------------------------------------%
%
%\begin{problem}{\it Documentation (5 pts)}
%
%
%\qq{(a)}2 Documentation (5 pts)
%One of the most valuable skills a software developer has is the ability to find, read, and understand documentation. The answers to these questions are not in your textbook, or in any official class documentation. However, this is not intended to be a puzzle - these answers are not hard to find, and you do not need to know about any additional documentation that we have not discussed with you in class. We would like you to get comfortable searching for answers in ways that might be unfamiliar for you.
%\dls What is the version of Java installed on athena?
%\dl How do you enable assertions when running a Java program?
%\dl How do you convert a double to a string in Java?
%\dl How do you create a jar file containing the files in directory {\tt asst1}?\dle
%You must include in your submission:
%[PDF] The answer to each question.
%[PDF] Where you found the information in order to receive full credit. This could be by issuing a shell command, using some GUI element, on the web at a specific URL, or on a manual page.
%[PDF] How long you spent on this question.
%	\qe
%
%\qq{(b)}
%	\qe 
%	
%\qq{(c)}
%	\qe 
%
%\qq{(d)}
%	\qe
%
%\qq{(e)}
%	\qe 
%	
%\qq{(f)}
%	\qe 
%
%  ____
% |__ /
%  |_ \
% |___/
%% PROBLEM 3 %-------------------------------------------------------------------------------%
%
%\begin{problem}{\it }
%
%
%\qq{(a)}
%	\qe
%
%\qq{(b)}
%	\qe 
%	
%\qq{(c)}
%	\qe 
%
%\qq{(d)}
%	\qe
%
%\qq{(e)}
%	\qe 
%	
%\qq{(f)}
%	\qe 
%
%  _ _  
% | | | 
% |_  _|
%   |_| 
%% PROBLEM 4 %-------------------------------------------------------------------------------%
%
%\begin{problem}{\it }
%
%
%\qq{(a)}
%	\qe
%
%\qq{(b)}
%	\qe 
%	
%\qq{(c)}
%	\qe 
%
%\qq{(d)}
%	\qe
%
%\qq{(e)}
%	\qe 
%	
%\qq{(f)}
%	\qe 
%
%
%
%  ___ 
% | __|
% |__ \
% |___/
%% PROBLEM 5 %-------------------------------------------------------------------------------%
%   __ 
%  / / 
% / _ \
% \___/
%%%%%%%%%%%%%%%%%%%%%%%%%%%%%%%%%%%%%%%%%%%%%%%%%%%%%%%%%%%%%%%%%%%%%%%%%%%%%%%%%%%%%%%%%%%%%%
%  ____ 
% |__  |
%   / / 
%  /_/  
%%%%%%%%%%%%%%%%%%%%%%%%%%%%%%%%%%%%%%%%%%%%%%%%%%%%%%%%%%%%%%%%%%%%%%%%%%%%%%%%%%%%%%%%%%%%%%
%  ___ 
% ( _ )
% / _ \
% \___/
%%%%%%%%%%%%%%%%%%%%%%%%%%%%%%%%%%%%%%%%%%%%%%%%%%%%%%%%%%%%%%%%%%%%%%%%%%%%%%%%%%%%%%%%%%%%%%
%  ___ 
% / _ \
% \_, /
%  /_/ 
%%%%%%%%%%%%%%%%%%%%%%%%%%%%%%%%%%%%%%%%%%%%%%%%%%%%%%%%%%%%%%%%%%%%%%%%%%%%%%%%%%%%%%%%%%%%%%
%   __  
%  /  \ 
% | () |
%  \__/ 
%%%%%%%%%%%%%%%%%%%%%%%%%%%%%%%%%%%%%%%%%%%%%%%%%%%%%%%%%%%%%%%%%%%%%%%%%%%%%%%%%%%%%%%%%%%%%%
%
%\begin{problem}{\it }
%
%\paragraph{(a)} 
%
%\subparagraph{(a)} 
%\subparagraph{(a)} 
%
%%%%%%%%%%%%%%%%%%%%%%%%%%%%%%%%%%%%%%%%%%%%%%%%%%%%%%%%%%%%%%%%%%%%%%%%%%%%%%%%%%%%%%%%%%%%%%
%\paragraph{(a)} 
%
%\paragraph{(a)} 
%
%\paragraph{(a)} 
%
%%%%%%%%%%%%%%%%%%%%%%%%%%%%%%%%%%%%%%%%%%%%%%%%%%%%%%%%%%%%%%%%%%%%%%%%%%%%%%%%%%%%%%%%%%%%%%
% % % % % % % % % % % % % % % % % % % % % % % % % % % % % % % % % % % % % % % % % % % % % % % 
%%%%%%%%%%%%%%%%%%%%%%%%%%%%%%%%%%%%%%%%%%%%%%%%%%%%%%%%%%%%%%%%%%%%%%%%%%%%%%%%%%%%%%%%%%%%%%
%\docend
%
%
%\xxx
%
%$\<\x\ra$
%
%%%%%BOX%%%%%%%%%%%%%%%%%%%%%%%%%%%%	
%%%% 							%%%%
%\rrb{							%%%%
%	 							%%%%
%								%%%%
%					 			%%%%
%		blah						%%%%
%								%%%%
%								%%%%
%}%%%					 		%%%%
%%%%%%%%%%%%%%%%%%%%%%%%%%%%%%%%%%%%
%
%%%%%BOX%%%%%%%%%%%%%%%%%%%%	
%%%% 					%%%%
%\pb{					%%%%
%	 					%%%%
%	blah					%%%%
%						%%%%
%						%%%%
%						%%%%
%						%%%%
%}%%%					%%%%
%%%%%%%%%%%%%%%%%%%%%%%%%%%%
%
%
%%%%%BOX%%%%%%%%%%%%
%%%% 			%%%%
%\brt{			%%%%
%	blah			%%%%
%				%%%%
%				%%%%
%				%%%%
%				%%%%
%				%%%%
%}%%%			%%%%
%%%%%%%%%%%%%%%%%%%%
%
%\begin{eqnarray*}
%x_2|x_1 &=_{def.}& \lnot(x_2=0)\land\exists x_3(x_2\times x_3=x_1)\\
%Prime(x_1) &=_{def.}& (1<x_1\land\fa x_2(x_2|x_1\to(x_2=x_1\lor x_2=1)))\\
%\end{eqnarray*}
% 
%\brt{Also unsure how to proceed, given my lack of clarity on 2(b) especially with the vaguely specified length ($k$) of the sequence.}
%\brt{\rrr{$NSeq(n,k)=_{def.}$ is true if your sequence is $k$ long.
%      }     }
%
%$Pair(n,a,b) =_{def.} n = 2^\land(a+1) \times 3^\land(b+1)$\\
%$\bk{Inc(n,m)=_{def.} \exists x_1(Prime(x_1)\land x_1^{m+1}|n)}$
%
%
%
%\renewcommand{\arraystretch}{1.5}
%
%
%   \begin{tabular}{ l | l | l  l  l l }
%    \hline
%	\hline
%$s^2$	&$$ &$$\\
%$s^1$	&$$ &$$\\
%$s^0$	&$$	&$$\\
%    \end{tabular}
%
%
%
%\Tree [.TP [.NP [.N [.N'       ] ] ]
%[.T' [.T                       ] 
%[.NegP [.Neg' [.Neg            ] 
%[.VP [.V' [.V                  ] 
%[.VP [.V' [.V                  ] 
%[.NP [.N' [.N                  ] ] ] ] ] ] ] ] ] ] ] ]
%
%\Tree [.TP [.NP [.N       ] ] 
%[.T' [.T                  ] 
%[.VP [.V' [.V             ] 
%[.NP [.N' [.N             ] ] ] ] ] ] ] 
%
%\Tree [.TP [.NP I ] 
%[.T' [.T haven't$_{k,n}$ ] 
%[.NegP [.Neg' [.$t_n$ ] 
%[.VP [.V' [.$t_k$ ] 
%\qroof{written the letter}.VP ] ] ] ] ] ] \\
%
%\Tree [.CP [.C Haven't$_{k,n} ] 
%[.TP [.NP I ] 
%[.T' [.T $t_{k,n}$ ] 
%[.NegP [.Neg' [.$t_n$ ] 
%[.VP [.V' [.$t_k$ ] 
%\qroof{written the letter}.VP ] ] ] ] ] ] ] \\
%
%
%\xxx
%
%$\approx \ll \gg \cap \cup \lor \land \lnot \times \equiv$
%{\color{red} 
%
%\ee{?} Which system?
%}
%
%$x(t) = e^{at}x(0)+\displaystyle\int_0^t e^{a(t-\tau)}bu(\tau)d\tau$
%
%
%$\mathcal{L}(\dot{x}(t)) = sX(s) - x|_{_{0^+}}$
%
%\boxed{$\mathcal{L}(\pp{x}(t)) = s^2X(s) - sx|_{_{0^+}} -\p{x}|_{_{0^+}}$}\\
%
%
%\begin{center}
%    \begin{tabular}{| l | l | l | l |}
%    \hline
%    Level & $\zeta$ & $\zeta\w_n$ & $\w_n$  \\ \hline
%    1 &  &  &  \\ \hline
%    2 &  &  &  \\ \hline
%    3 &  &  &  \\ \hline
%
%    \hline
%    \end{tabular}
%\end{center}
%
%M = \begin{bmatrix}
%       0 & 0 & 0           \\
%       0 & 0 & 0 \\
%       0 & 0 & 0
%     \end{bmatrix}
%
% \begin{bmatrix}
%       0 & 0          \\
%       0 & 0 
%     \end{bmatrix}
% \begin{bmatrix}
%       0            \\
%       0 
%     \end{bmatrix} = 
% \begin{bmatrix}
%       0            \\
%       0 
%     \end{bmatrix}
%
%
%\begin{center}
%\includegraphics[width=9cm]{pset-2-sketch.JPG}
%\end{center}
%
%\e \textit{(.ps plots attached along with model .slx file in zip)}\\
%\hrule
%
%\e Use the data collected to plot the output of the right motor encoder. \\[1ex]
%Hint: ”Compare to constant” block can be used to check the signal levels \\[1ex]
%
%\e $\vec{r} = R\cos{(b_0t)}\vec{i} + \vec{j} +\vec{k}$
%\e $\vec{\p{r}} = \vec{i} + \vec{j} +\vec{k}$
%\e $\vec{\pp{r}} = \vec{i} + \vec{j} +\vec{k}$
%
%\paragraph{(B)} 
%
%\e $\vec{r} = \vec{e_r} + \vec{e_\h} +\vec{k}$
%\e $\vec{\p{r}} = \vec{e_r} + \vec{e_\h} +\vec{k}$
%\e $\vec{\pp{r}} = \vec{e_r} + \vec{e_\h} +\vec{k}$
%
%
%\paragraph{(C)} 
%
%\e $\vec{r} = \vec{e_t} + \vec{e_n} +\vec{e_b}$
%\e $\vec{\p{r}} = \vec{e_t} + \vec{e_n} +\vec{e_b}$
%\e $\vec{\pp{r}} = \vec{e_t} + \vec{e_n} +\vec{e_b}$
%
%\begin{multicols}{3}
%
%\ee{1} \bb{What points are on the root locus?}\\
%
%$\phi_c(s) = 1 + KL_d(s)$
%
%$L_d(s) = -\f{1}{K}$
%\\
%\ee{2} \bb{Where does the root locus start?}\\
%
%$\phi_c = D_cD_p+KN_cN_p = 0$
%
%$K \rightarrow 0$ so $D_cD_p = 0$ (poles of plant and compensator)
%\\
%\ee{3} \bb{Where does the root locus end?}\\
%
%$m$ poles head to the $m$ finite zeros of $L_d(s)$
%
%remaining $n-m$ poles go to $|s| = \infty$ along asymptotes defined by
% 
%$\phi_l = \f{180\dg + 360\dg \cdot(l-1)}{n-m} \;\;\;l = 1,2,...,(n-m)$
%
%centroid of asymptotes $\alpha = \f{\sum^n p_j - \sum^m z_i}{n-m} $\\
%
%\\
%\ee{4} \bb{When/where is the locus on the real line?}\\
%to the left of an odd number of real axis
%poles and zeros $(K>0)$
%\\
%\ee{5} \bb{What are the break-away and break-in points on the real line?}\\
%
%\psi
%\Psi
%
%\begin{spacing}{2.0} 
%\end{spacing}
%
%{\Large\bf\noindent References}\\
%
%
%
%\noindent \hangindent=1cm H\'{a}jek, Alan. "Interpretations of Probability." \ii{Stanford Encyclopedia of Philosophy}. Stanford, CA: Stanford University Metaphysics Research Lab, 2011\\
%
%\noindent \hangindent=1cm White, Roger. "Problem Set 6." \ii{24.280 Foundations of Probability}. Cambridge, MA: Massachusetts Institute of Technology, 2013.
%
%

\end{document}